\documentclass[8pt]{article}
\usepackage[english]{babel}
\usepackage[T1]{fontenc}
\usepackage{parskip}
\usepackage{setspace}
\usepackage{helvet}
\usepackage[margin=1in]{geometry}
\usepackage[bottom]{footmisc}
\newcommand{\q}[1]{``#1''}
\usepackage{fancyhdr}
\pagestyle{fancy}
\fancyhf{}
\rhead{\thepage}
\lhead{ECE 290: Case \#4, m9lai, yx6li, ba3brown p23huang}
\doublespace
\renewcommand{\familydefault}{\sfdefault}


\begin{document}
\begin{center}\section*{Case \#4 - Thalidomide Drug Consequences}\end{center}

\subsection{The ethically ambiguous situation} 

We will be examining the Thalidomide drug catastrophe in this case study. Thalidomide was available to the Canadian public as a drug as early as 1959, and officially licensed for presciption use on April 1, 1961\footnote{\q{The Canadian Tragedy.} - Thalidomide. N.p., n.d. Web. 17 Mar. 2016.}. Thalidomide was a drug that was marketed towards pregnant woman as a sedative that reduced the effects of morning sickness. Morning sickness is a condition affecting pregnant woman during the first trimester of pregnancy, which has symptoms such as nausea and vomiting.\footnote{\q{Morning Sickness During Pregnancy.} American Pregnancy. N.p., n.d. Web. 17 Mar. 2016.}. Thalidomide had disastorous side effects, such as causing peripheral neuritis, killing thousands of babies and causing severe birth defects\textsuperscript{1}.  \par

The ethically ambiguous situation arises from the Canadian government allowing the drug to be sold and moreover the extensive amount of time it took for the drug to be pulled off the market even after the side effects were discovered; the drug was pulled from the West German and United Kingdom markets by December 2, 1961, but remained legally available in Canada until March 2, 1962\textsuperscript{1}. \par

Furthermore, in Canada, families who were victims of the drug were forced to settle out-of-court and no overarching trial verdict was ever conclusively reached\textsuperscript{1}. This is in direct contrast with foreign countries such as the UK and West Germany that successfully held class action law suits to seek compensation for the victims of thalidomide\textsuperscript{1}. This brings into discussion the legal and political ambiguities surrounding how justice and tort laws were applied to the thalidomide catastrophe.   \par


\subsection{The stakeholders and their importance} 
\begin{itemize}
\item Victims and their families

Victims were essential stakeholders in this case as they were the ones ultimately affected by the side effects of thalidomide. Their families are equally as important as they would be naturally concerned about the health and well-being of their children.

\item Doctors and Pharmacists

Doctors and pharmacists played a key role in this case as they were the ones who would have prescribed the thalidomide to women going through pregnancy. As a doctor, you would want to ensure the best possible care of your patients and would be very concerned if you found out the drug you had prescribed had such dangerous side effects.

\item Chemie Gr\"{u}nenthal Management/Executives

Chemie Gr\"{u}nenthal is another major stakeholder as they manufactured and sold the drug. They effectively promised that thalidomide would be a safe, effective drug which was far off from reality.
\end{itemize}

\subsection{The perspectives of stakeholders}  
\begin{itemize}

\item Perspectives of Victims and their families

\begin{enumerate} 
\item Naturally, most users of thalidomide trusted the drug since it was licensed for prescription use and referred to as the \q{wonder drug}.\footnote{\q{Thalidomide: A Canadian Tragedy}. http://www.thalidomide.ca/. Thalidomide Victims Association of Canada., 2016. Web. 17 Mar. 2016.} Pregnant woman who experienced morning sickness and wanted to alleviate their symptoms would see the drug as a viable option.
\item 
On the other hand, some individuals may have been more skeptical about new medicine since side effects are not well known. These people would choose not to use the drug right after its release.
\end{enumerate}

\item Perspectives of Doctors/Pharmacists

\begin{enumerate} 
\item A popular drug like thalidomide serves as a good source of income for pharmacists, so pharmacists might promote this drug for their personal gain. At the same time, doctors or pharmacists who believed in the effectiveness of the drug may genuinely want to help their patients, so they would promote it as well.
\item Other doctors like Frances Kelsey insisted on gathering more data before selling the drug to the public. Doctors like her were against the rushed delivery of thalidomide.
\end{enumerate}

\item Perspectives of Chemie Gr\"{u}nenthal Management \& Executives

\begin{enumerate}
\item Since thalidomide was discovered to be very effective in treating morning sickness and other conditions, it would generate a large revenue for Chemie Gr\"{u}nenthal if it was made available to the public and advertized. From a business standpoint, it made sense for Chemie Gr\"{u}nenthal to release the drug as soon as possible, which means that it cannot be thoroughly tested to the standards in which it should be.
\item As a pharmaceutical company, it is the company's duty to deliver safe medication to the public. Some employees of Chemie Gr\"{u}nenthal may have argued against the rushed release of the drug.
\end{enumerate}
\end{itemize}
\subsection{Resolution of the ethical situation}

Even though thalidomide was eventually discovered to be a dangerous drug, by the time it was banned in most countries, \q{between ten and twenty thousand babies born disabled as a consequence of the drug thalidomide.}\footnote{Crane, Rachel. \q{Thalidomide: A Canadian Tragedy}. http://www.thalidomide.ca/. Thalidomide Victims Association of Canada., 2016. Web. 17 Mar. 2016.} Critical stateholders such as Chemie Gr\"{u}nenthal could have conducted more extensive clinical trials to discover any additional side affects of the drug. In addition the Canadian Government could have done additional testing on any drugs that will be sold in Canada to prevent such tragedies from occurring.
\par
As a response to the ethical issues raised by the use of thalidomide, new regulations regarding drugs and medication were put into effect. Two groups were formed as a result of this tragedy. The first was The Thalidomide Victims Association of Canada. Their goal is to \q{to empower [thaliodomide victims] and to improve their quality of life}.\footnote{Crane, Rachel. \q{TVAC and Its Mission}. http://www.thalidomide.ca/. Thalidomide Victims Association of Canada., 2016. Web. 17 Mar. 2016.} Another organization created is The Society of Toxicology of Canada. Their mission is to raise awareness on the potential side effects of drugs and to enforce the Conservation Environment Protection Act, which include the new regulations surrounding drug testing brought forth after the thalidomide tragedy to prevent similar events from happening in the future.\footnote{\q{Online sources of toxicological information in Canada}. http://www.sciencedirect.com/. Science Direct., 2016. Web. 17 Mar. 2016.}

\subsection{Our views on the resolution of the ethical situation}
Overall our group found that the resolution of situation was unsatisfactory. While thalidomide was ultimately taken off the shelves and new regulations were put in place, we believe the situation could have been handled a lot better in Canada. For instance, thalidomide was not fully taken off the shelves until nearly half a year after Germany and the UK had the drug withdrawn. In addition, in other countries the victims of the drug and their families were able to enter into class action legal suits against the drug companies who manufactured or distributed the drug, while in Canada families were forced to go at it alone. Often this required them to settle out of court for undisclosed amounts instead of being able to recieve monthly or annual payments based on the level of disability of the victim.\footnote{Crane, Rachel. \q{Thalidomide: A Canadian Tragedy}. http://www.thalidomide.ca/. Thalidomide Victims Association of Canada., 2016. Web. 17 Mar. 2016.} Thus we believe a more satisfactory resolution would have been to have the drug withdrawn immediately upon hearing the reports from foreign countries. Thalidomide also should have been tested signficantly before approving its use inside of Canada.




\end{document}