\documentclass{article}
\usepackage[english]{babel}
\usepackage[T1]{fontenc}
\usepackage{parskip}
\usepackage{hyperref}
\usepackage{setspace}
\usepackage[margin=1in]{geometry}
\usepackage[bottom]{footmisc}
\newcommand{\q}[1]{``#1''}
\usepackage{fancyhdr}

\pagestyle{fancy}
\fancyhf{}
\rhead{\thepage}
\lhead{ECE 290: Case \#2, m9lai, yx6li, ba3brown p23huang}
\doublespace
\renewcommand{\familydefault}{\sfdefault}


\begin{document}
\begin{center}\section*{Case \#2 - Space Shuttle Challenger Disaster}\end{center}

\subsection{Ethically ambiguous situations} 

We will be examining the Space Shuttle Challenger Disaster of 1986 where \q{the Challenger and its seven-member crew were lost 73 seconds after launch when a booster failure resulted in the breakup of the vehicle}.\footnote{Dumoulin, Jim. \q{Shuttle Orbiter Challenger (OV-99)}. Science.ksc.nasa.gov. N.p., 2016. Web. 30 Jan. 2016} The incident was the result of several critical oversights that involved political officials, NASA and Morton Thiokol management as well as engineers. The case also involves several ethically ambiguous situations, where clear decision making would have been affected by complications that were occuring at the time. \par 

One example of an ambiguous situation in the Space Shuttle Challenger Distaster was NASA having to choose between safety and productivity. Several months prior to the Challenger launch, NASA was under extreme pressure to launch in a timely manner. \q{Political pressure to provide a reliable, reusable space vehicle with rapid turn around time} would have \q{seriously hindered the ability for effective systems integration and development}.\footnote{Forrest, Jeff. \q{Space Shuttle Challenger Disaster 1986}. Dssresources.com. N.p., 2016. Web. 31 Jan. 2016.} This created an ethically ambiguous situation for NASA: When a safety concern arises, do you spend additional time to investigate it or do you continue building the spacecraft in order to not miss deadlines? As seen with challenger, the outcome can be catastrophic if critical safety concerns are not addressed properly. However, it is also important to note that \q{if the Shuttle never launches, NASA fails its mission in equal measure as it does when it has accidents}.\footnote{Vaughan, Diane. \q{NASA Cultures}. history.nasa.gov. N.p., 2016. Web. 31 Jan. 2016.} This certainly would have placed huge amounts of stress on NASA, resulting in tough decisions leading up to the launch and an overall ambiguous situation. \par

Another ethically ambigious situation occured with groupthink and whistleblowing, when the \q{night before the fateful Challenger launch, there was a teleconference between engineers and officials at Morton Thiokol, Kennedy Space Center and Marshall Space Flight Center}.\footnote{Hall, Joseph. \q{Columbia And Challenger: Organizational Failure At NASA}. sciencedirect.com. N.p., 2016. Web. 31 Jan. 2016.} A group of low level engineers at Thiokol expressed their concern of the possibility of a critical failure of the O-rings in cold temperatures, but they were consequently shut down by NASA officials. A shuttle program manager Lawrence Mulloy said in the meeting: \q{My God, Thiokol, when do you want me to launch - next April?}\footnote{Berkes, Howard. \q{Remembering Roger Boisjoly:  He Tried To Stop Shuttle Challenger Launch}. NPR.org. N.p., 2012. Web. 31 Jan. 2016.} This type of groupthink would have led to an ambiguous situation in which the Thiokol engineers could have attempted whistleblowing (i.e calling management out on their choice to prioritize meeting deadlines over meeting safety requirements), but this would have come at the risk of stepping out line and appearing incompetent. Since the launch date had already been pushed back several days prior, the engineers at Thiokol had a very difficult choice to make: Should they speak up against their higher ups or accept their management's decision? Overall, the meeting the night before the Challenger launch was a common example of \q{go-fever} (being in a rush to finish a project while overlooking potential faults), and an ethically ambiguous situation for the engineers of Morton Thiokol. 
\par

\subsection{Importance of stakeholders} 
\begin{itemize}

\item
NASA managers and scientists

NASA managers and scientists played a crucial role in the Space Shuttle Challenger Disaster as they were the ones overseeing the whole project. Since NASA scientists were trying to ensure the technology worked succesfully and NASA managers were trying to ensure a specific launch date was met, they both played a huge role in the outcome of the challenger incident. \par
\vspace{10pt}
\item
American Government

The American Government, or the Reagan adminstration, also played a very large role in the Challenger incident. In addition to funding NASA financially, they were also interested in seeing the space shuttle completed as soon as possible in order to inspire American citizens as well as the rest of the world. President Ronald Reagan even started the Teachers in Space Program, a program where a teacher was supposed to be first citizen passenger in the history of space travel. In that way, Reagan said, \q{All of America will be reminded of the crucial role that teachers and education play in the life of our nation. I can’t think of a better lesson for our children and our country.}\footnote{Citizens In Space,. \q{Teachers In Space: The History}. N.p., 2012. Web. 31 Jan. 2016.} Reagan adminstration's desire to inspire the lives of everyday people through the Challenger mission demonstrates their importance as a stakeholder.

\vspace{10pt}
\item
Morton Thiokol Managers and Engineers

Morton Thiokol played a huge role in the Space Shuttle Challenger Disaster as they were the ones who manufactured the SRB's (Solid Rocket Boosters), which contained a fatal design flaw in the O-rings that ultimately led to the destruction of the Shuttle. In addition, Thiokol engineers tried to warn NASA against the dangers of launching the Space Shuttle in low temperatures on several occasions. Roger Boisjoly, a booster rocket engineer at Thiokol even sent a memo to the company's vice president several months prior to the incident. \footenote{Webcitation.org,. \q{Webcite Query Result}. N.p., 2016. Web. 31 Jan. 2016.} Both the company's role in designing the SRB's as well as the Thiokol engineers who tried to prevent the Challenger disastor made Morton Thiokol an extremely important stakeholder in the events that followed. 

\end{itemize}
\vspace{5pt}
\subsection{Resolution of the ethical situation}
As a response to the ethical issues raised by the application of EITs, there were some investigations done into the nature of these of these activities by the CIA. In response to the ethical issues raised by how terror suspects should be treated, the Detainee Treatment Act of 2005 was passed into law; \q{It prohibits the \q{cruel, inhuman, or degrading treatment or punishment} of detainees and provides for \q{uniform standards} for interrogation}.\footnote{\q{Detainee Treatment Act of 2005 (H.R. 2863, Title X).} CFR.org. Council on Foreign Relations, 30 Dec. 2005. Web} However, the bill was criticized because it failed to provide concrete definitions of what was acceptable and what was not. This greatly hindered any attempt to objectively discuss the situation or hold any party accountable for what occurred. In 2008 the Senate tried to pass an act to clarify the 2005 bill with more specific wording, but it was veteod by Bush after passing through the other stages of government.\footnote{Beech, Eric. \q{Bush Vetoes Bill Outlawing CIA Waterboarding.} Reuters. N.p., 8 Mar. 2008. Web.} \par
\vspace{8pt}
In 2014, 10\% of the Senate Intelligence Committee Report on CIA torture investigating the use of EITs was released to the public. It stated that: \q{The CIA's use of its enhanced interrogation techniques was not an effective means of acquiring intelligence or gaining cooperation from detainees}. Moreover, At least 26 of the 119 prisoners (22\%) held by the CIA were subsequently found by the CIA to have been improperly detained.  \footnote{\q{Committee Study of the Central Intelligence Agency's Detention and Interrogation Program} Senate Select Committee on Intelligence. Dec 2014. United States Senate.} \par
\pagebreak
\vspace{10pt}
\subsection{Our views on the resolution of the ethical situation}

Overall, we found the resolution to be unsatisfactory and that the EIT program was unethical. Although a major argument for the application of EITs is that more extreme measures are required to protect national security and prevent another 9/11 from occurring, the Senate Intelligence Committee Report noted that EITs were ineffective, and 22\% of prisoners were improperly detained and hence the EITs were used on innocent people. \par
\vspace{5pt}
From the perspective of the stakeholders, we can also examine this ethically ambiguous situation. Physicians were required to infringe on a major axiom in medical ethics, and the programs they developed for the CIA helped facilitate the torture of terror suspects. This is especially significant because for a profession, whether engineering or medicine, years of schooling are required to formally join the profession because of the impact these professions can have on society and life. To challenge such a code of ethics is indeed challenging the very basis of a profession itself. If it is seen as acceptable to bend the rules set down by professional ethics, then disastorous consequences can arise. In the perspective of prisoners, the EITs infringed upon their human rights with little justification - many of them were innocent or improperly detained, and the CIA report noted that the EITs had limited effectiveness in acquiring useful information. In the perspective of U.S citizens, they had much of this information hidden from them until more than 10 years later; they were not empowered to critically examine the situation and the ethics surrounding it, even if they are the ones that the government is supposed to represent and serve. \par

Furthermore, we must recognize there were virtually no legal repercussions faced by the CIA in employing EITs. This could be seen as a precedence for similar situations in the future --- that the government has the power to enact drastic measures and make unethical decisions without transparency to U.S citizens. \par
\vspace{5pt}

Another major aspect of discussion is that one of the stakeholders, President Bush, was able to veto a 2008 bill intending to more specifically define the ethical boundaries for CIA activities missing from the Detainee Treatment Act of 2005. By doing so, the only provisions in place against EITs are ambiguously defined in the 2005 bill. By our definition of what is ethical - a decision that satisfies as many stakeholders as possible, President Bush\lq s veto visibly served only the interests of the CIA and himself, and ultimately the perspectives of the other stakeholders were not accounted for. Hence, our conclusion regarding this case on EITs is that we find that the application of EITs were unnecessary, ineffective and unethical. \par

\pagebreak
\begin{center}\section*{References}\end{center}

Bush Vetoes Bill Outlawing CIA Waterboarding. Reuters. N.p., 8 Mar. 2008. Web. \par

\vspace{5pt}
Committee Study of the Central Intelligence Agency's Detention and Interrogation Program Senate. Select Committee on Intelligence. Dec 2014. United States Senate. \par

\vspace{5pt}
Detainee Treatment Act of 2005 (H.R. 2863, Title X). CFR.org. Council on Foreign Relations, 30 Dec. 2005. Web \par
\vspace{5pt}

Physicians for Human Rights. \q{Experiments in Torture: Evidence of Human Subject Research and Experimentation in the \q{Enhanced} Interrogation Program} The Torture Papers. Jun 2010. PHR.\par
\vspace{5pt}

Smith, C. M. (2005), Origin and Uses of Primum Non Nocere---Above All, Do No Harm!. Journal of Clinical Pharma, 45: 371---377. doi: 10.1177/0091270004273680\par




\end{document}